%% Generated by Sphinx.
\def\sphinxdocclass{report}
\documentclass[letterpaper,10pt,russian]{sphinxmanual}
\ifdefined\pdfpxdimen
   \let\sphinxpxdimen\pdfpxdimen\else\newdimen\sphinxpxdimen
\fi \sphinxpxdimen=.75bp\relax
\ifdefined\pdfimageresolution
    \pdfimageresolution= \numexpr \dimexpr1in\relax/\sphinxpxdimen\relax
\fi
%% let collapsible pdf bookmarks panel have high depth per default
\PassOptionsToPackage{bookmarksdepth=5}{hyperref}

\PassOptionsToPackage{booktabs}{sphinx}
\PassOptionsToPackage{colorrows}{sphinx}

\PassOptionsToPackage{warn}{textcomp}
\usepackage[utf8]{inputenc}
\ifdefined\DeclareUnicodeCharacter
% support both utf8 and utf8x syntaxes
  \ifdefined\DeclareUnicodeCharacterAsOptional
    \def\sphinxDUC#1{\DeclareUnicodeCharacter{"#1}}
  \else
    \let\sphinxDUC\DeclareUnicodeCharacter
  \fi
  \sphinxDUC{00A0}{\nobreakspace}
  \sphinxDUC{2500}{\sphinxunichar{2500}}
  \sphinxDUC{2502}{\sphinxunichar{2502}}
  \sphinxDUC{2514}{\sphinxunichar{2514}}
  \sphinxDUC{251C}{\sphinxunichar{251C}}
  \sphinxDUC{2572}{\textbackslash}
\fi
\usepackage{cmap}
\usepackage[T1]{fontenc}
\usepackage{amsmath,amssymb,amstext}
\usepackage{babel}





\usepackage[Sonny]{fncychap}
\ChNameVar{\Large\normalfont\sffamily}
\ChTitleVar{\Large\normalfont\sffamily}
\usepackage{sphinx}

\fvset{fontsize=auto}
\usepackage{geometry}


% Include hyperref last.
\usepackage{hyperref}
% Fix anchor placement for figures with captions.
\usepackage{hypcap}% it must be loaded after hyperref.
% Set up styles of URL: it should be placed after hyperref.
\urlstyle{same}

\addto\captionsrussian{\renewcommand{\contentsname}{Contents:}}

\usepackage{sphinxmessages}
\setcounter{tocdepth}{1}



\title{Calculator}
\date{мар. 28, 2025}
\release{1.0}
\author{Ilya}
\newcommand{\sphinxlogo}{\vbox{}}
\renewcommand{\releasename}{Выпуск}
\makeindex
\begin{document}

\ifdefined\shorthandoff
  \ifnum\catcode`\=\string=\active\shorthandoff{=}\fi
  \ifnum\catcode`\"=\active\shorthandoff{"}\fi
\fi

\pagestyle{empty}
\sphinxmaketitle
\pagestyle{plain}
\sphinxtableofcontents
\pagestyle{normal}
\phantomsection\label{\detokenize{index::doc}}


\sphinxAtStartPar
Add your content using \sphinxcode{\sphinxupquote{reStructuredText}} syntax. See the
\sphinxhref{https://www.sphinx-doc.org/en/master/usage/restructuredtext/index.html}{reStructuredText}
documentation for details.

\sphinxstepscope


\chapter{sphinx}
\label{\detokenize{modules:sphinx}}\label{\detokenize{modules::doc}}
\sphinxstepscope


\section{Calculator module}
\label{\detokenize{Calculator:module-Calculator}}\label{\detokenize{Calculator:calculator-module}}\label{\detokenize{Calculator::doc}}\index{module@\spxentry{module}!Calculator@\spxentry{Calculator}}\index{Calculator@\spxentry{Calculator}!module@\spxentry{module}}
\sphinxAtStartPar
Модуль calculator.py

\sphinxAtStartPar
Этот модуль предоставляет базовые арифметические операции и демонстрирует
документирование кода для Sphinx.
\index{Calculator (класс в Calculator)@\spxentry{Calculator}\spxextra{класс в Calculator}}

\begin{fulllineitems}
\phantomsection\label{\detokenize{Calculator:Calculator.Calculator}}
\pysigstartsignatures
\pysigline
{\sphinxbfcode{\sphinxupquote{\DUrole{k}{class}\DUrole{w}{ }}}\sphinxcode{\sphinxupquote{Calculator.}}\sphinxbfcode{\sphinxupquote{Calculator}}}
\pysigstopsignatures
\sphinxAtStartPar
Базовые классы: \sphinxcode{\sphinxupquote{object}}

\sphinxAtStartPar
Класс Calculator предоставляет методы для выполнения арифметических операций
с сохранением истории вычислений.
\begin{quote}\begin{description}
\sphinxlineitem{Переменные}
\sphinxAtStartPar
\sphinxstyleliteralstrong{\sphinxupquote{history}} (\sphinxstyleliteralemphasis{\sphinxupquote{list}}\sphinxstyleliteralemphasis{\sphinxupquote{{[}}}\sphinxstyleliteralemphasis{\sphinxupquote{str}}\sphinxstyleliteralemphasis{\sphinxupquote{{]}}}) \textendash{} Список для хранения истории операций

\end{description}\end{quote}
\index{add() (метод Calculator.Calculator)@\spxentry{add()}\spxextra{метод Calculator.Calculator}}

\begin{fulllineitems}
\phantomsection\label{\detokenize{Calculator:Calculator.Calculator.add}}
\pysigstartsignatures
\pysiglinewithargsret
{\sphinxbfcode{\sphinxupquote{add}}}
{\sphinxparam{\DUrole{n}{a}\DUrole{p}{:}\DUrole{w}{ }\DUrole{n}{float}}\sphinxparamcomma \sphinxparam{\DUrole{n}{b}\DUrole{p}{:}\DUrole{w}{ }\DUrole{n}{float}}}
{{ $\rightarrow$ float}}
\pysigstopsignatures
\sphinxAtStartPar
Складывает два числа, сохраняет операцию в истории и возвращает результат.
\begin{quote}\begin{description}
\sphinxlineitem{Параметры}\begin{itemize}
\item {} 
\sphinxAtStartPar
\sphinxstyleliteralstrong{\sphinxupquote{a}} (\sphinxstyleliteralemphasis{\sphinxupquote{float}}) \textendash{} Первое слагаемое

\item {} 
\sphinxAtStartPar
\sphinxstyleliteralstrong{\sphinxupquote{b}} (\sphinxstyleliteralemphasis{\sphinxupquote{float}}) \textendash{} Второе слагаемое

\end{itemize}

\sphinxlineitem{Результат}
\sphinxAtStartPar
Сумма a и b

\sphinxlineitem{Тип результата}
\sphinxAtStartPar
float

\end{description}\end{quote}

\end{fulllineitems}

\index{get\_history() (метод Calculator.Calculator)@\spxentry{get\_history()}\spxextra{метод Calculator.Calculator}}

\begin{fulllineitems}
\phantomsection\label{\detokenize{Calculator:Calculator.Calculator.get_history}}
\pysigstartsignatures
\pysiglinewithargsret
{\sphinxbfcode{\sphinxupquote{get\_history}}}
{}
{{ $\rightarrow$ list\DUrole{p}{{[}}str\DUrole{p}{{]}}}}
\pysigstopsignatures
\sphinxAtStartPar
Возвращает историю операций калькулятора.
\begin{quote}\begin{description}
\sphinxlineitem{Результат}
\sphinxAtStartPar
Список выполненных операций

\sphinxlineitem{Тип результата}
\sphinxAtStartPar
list{[}str{]}

\end{description}\end{quote}

\end{fulllineitems}


\end{fulllineitems}

\index{add() (в модуле Calculator)@\spxentry{add()}\spxextra{в модуле Calculator}}

\begin{fulllineitems}
\phantomsection\label{\detokenize{Calculator:Calculator.add}}
\pysigstartsignatures
\pysiglinewithargsret
{\sphinxcode{\sphinxupquote{Calculator.}}\sphinxbfcode{\sphinxupquote{add}}}
{\sphinxparam{\DUrole{n}{a}\DUrole{p}{:}\DUrole{w}{ }\DUrole{n}{float}}\sphinxparamcomma \sphinxparam{\DUrole{n}{b}\DUrole{p}{:}\DUrole{w}{ }\DUrole{n}{float}}}
{{ $\rightarrow$ float}}
\pysigstopsignatures
\sphinxAtStartPar
Складывает два числа и возвращает результат.
\begin{quote}\begin{description}
\sphinxlineitem{Параметры}\begin{itemize}
\item {} 
\sphinxAtStartPar
\sphinxstyleliteralstrong{\sphinxupquote{a}} (\sphinxstyleliteralemphasis{\sphinxupquote{float}}) \textendash{} Первое слагаемое

\item {} 
\sphinxAtStartPar
\sphinxstyleliteralstrong{\sphinxupquote{b}} (\sphinxstyleliteralemphasis{\sphinxupquote{float}}) \textendash{} Второе слагаемое

\end{itemize}

\sphinxlineitem{Результат}
\sphinxAtStartPar
Сумма a и b

\sphinxlineitem{Тип результата}
\sphinxAtStartPar
float

\end{description}\end{quote}
\begin{description}
\sphinxlineitem{Пример использования:}
\begin{sphinxVerbatim}[commandchars=\\\{\}]
\PYG{g+gp}{\PYGZgt{}\PYGZgt{}\PYGZgt{} }\PYG{n}{add}\PYG{p}{(}\PYG{l+m+mi}{2}\PYG{p}{,} \PYG{l+m+mi}{3}\PYG{p}{)}
\PYG{g+go}{5.0}
\end{sphinxVerbatim}

\end{description}

\end{fulllineitems}

\index{divide() (в модуле Calculator)@\spxentry{divide()}\spxextra{в модуле Calculator}}

\begin{fulllineitems}
\phantomsection\label{\detokenize{Calculator:Calculator.divide}}
\pysigstartsignatures
\pysiglinewithargsret
{\sphinxcode{\sphinxupquote{Calculator.}}\sphinxbfcode{\sphinxupquote{divide}}}
{\sphinxparam{\DUrole{n}{a}\DUrole{p}{:}\DUrole{w}{ }\DUrole{n}{float}}\sphinxparamcomma \sphinxparam{\DUrole{n}{b}\DUrole{p}{:}\DUrole{w}{ }\DUrole{n}{float}}}
{{ $\rightarrow$ float}}
\pysigstopsignatures
\sphinxAtStartPar
Делит первое число на второе и возвращает результат.
\begin{quote}\begin{description}
\sphinxlineitem{Параметры}\begin{itemize}
\item {} 
\sphinxAtStartPar
\sphinxstyleliteralstrong{\sphinxupquote{a}} (\sphinxstyleliteralemphasis{\sphinxupquote{float}}) \textendash{} Делимое

\item {} 
\sphinxAtStartPar
\sphinxstyleliteralstrong{\sphinxupquote{b}} (\sphinxstyleliteralemphasis{\sphinxupquote{float}}) \textendash{} Делитель (не должен быть нулем)

\end{itemize}

\sphinxlineitem{Результат}
\sphinxAtStartPar
Частное a и b

\sphinxlineitem{Тип результата}
\sphinxAtStartPar
float

\sphinxlineitem{Исключение}
\sphinxAtStartPar
\sphinxstyleliteralstrong{\sphinxupquote{ValueError}} \textendash{} Если b равно нулю

\end{description}\end{quote}
\begin{description}
\sphinxlineitem{Пример использования:}
\begin{sphinxVerbatim}[commandchars=\\\{\}]
\PYG{g+gp}{\PYGZgt{}\PYGZgt{}\PYGZgt{} }\PYG{n}{divide}\PYG{p}{(}\PYG{l+m+mi}{6}\PYG{p}{,} \PYG{l+m+mi}{3}\PYG{p}{)}
\PYG{g+go}{2.0}
\end{sphinxVerbatim}

\end{description}

\end{fulllineitems}

\index{multiply() (в модуле Calculator)@\spxentry{multiply()}\spxextra{в модуле Calculator}}

\begin{fulllineitems}
\phantomsection\label{\detokenize{Calculator:Calculator.multiply}}
\pysigstartsignatures
\pysiglinewithargsret
{\sphinxcode{\sphinxupquote{Calculator.}}\sphinxbfcode{\sphinxupquote{multiply}}}
{\sphinxparam{\DUrole{n}{a}\DUrole{p}{:}\DUrole{w}{ }\DUrole{n}{float}}\sphinxparamcomma \sphinxparam{\DUrole{n}{b}\DUrole{p}{:}\DUrole{w}{ }\DUrole{n}{float}}}
{{ $\rightarrow$ float}}
\pysigstopsignatures
\sphinxAtStartPar
Умножает два числа и возвращает результат.
\begin{quote}\begin{description}
\sphinxlineitem{Параметры}\begin{itemize}
\item {} 
\sphinxAtStartPar
\sphinxstyleliteralstrong{\sphinxupquote{a}} (\sphinxstyleliteralemphasis{\sphinxupquote{float}}) \textendash{} Первый множитель

\item {} 
\sphinxAtStartPar
\sphinxstyleliteralstrong{\sphinxupquote{b}} (\sphinxstyleliteralemphasis{\sphinxupquote{float}}) \textendash{} Второй множитель

\end{itemize}

\sphinxlineitem{Результат}
\sphinxAtStartPar
Произведение a и b

\sphinxlineitem{Тип результата}
\sphinxAtStartPar
float

\end{description}\end{quote}
\begin{description}
\sphinxlineitem{Пример использования:}
\begin{sphinxVerbatim}[commandchars=\\\{\}]
\PYG{g+gp}{\PYGZgt{}\PYGZgt{}\PYGZgt{} }\PYG{n}{multiply}\PYG{p}{(}\PYG{l+m+mi}{2}\PYG{p}{,} \PYG{l+m+mi}{3}\PYG{p}{)}
\PYG{g+go}{6.0}
\end{sphinxVerbatim}

\end{description}

\end{fulllineitems}

\index{subtract() (в модуле Calculator)@\spxentry{subtract()}\spxextra{в модуле Calculator}}

\begin{fulllineitems}
\phantomsection\label{\detokenize{Calculator:Calculator.subtract}}
\pysigstartsignatures
\pysiglinewithargsret
{\sphinxcode{\sphinxupquote{Calculator.}}\sphinxbfcode{\sphinxupquote{subtract}}}
{\sphinxparam{\DUrole{n}{a}\DUrole{p}{:}\DUrole{w}{ }\DUrole{n}{float}}\sphinxparamcomma \sphinxparam{\DUrole{n}{b}\DUrole{p}{:}\DUrole{w}{ }\DUrole{n}{float}}}
{{ $\rightarrow$ float}}
\pysigstopsignatures
\sphinxAtStartPar
Вычитает второе число из первого и возвращает результат.
\begin{quote}\begin{description}
\sphinxlineitem{Параметры}\begin{itemize}
\item {} 
\sphinxAtStartPar
\sphinxstyleliteralstrong{\sphinxupquote{a}} (\sphinxstyleliteralemphasis{\sphinxupquote{float}}) \textendash{} Уменьшаемое

\item {} 
\sphinxAtStartPar
\sphinxstyleliteralstrong{\sphinxupquote{b}} (\sphinxstyleliteralemphasis{\sphinxupquote{float}}) \textendash{} Вычитаемое

\end{itemize}

\sphinxlineitem{Результат}
\sphinxAtStartPar
Разность a и b

\sphinxlineitem{Тип результата}
\sphinxAtStartPar
float

\end{description}\end{quote}
\begin{description}
\sphinxlineitem{Пример использования:}
\begin{sphinxVerbatim}[commandchars=\\\{\}]
\PYG{g+gp}{\PYGZgt{}\PYGZgt{}\PYGZgt{} }\PYG{n}{subtract}\PYG{p}{(}\PYG{l+m+mi}{5}\PYG{p}{,} \PYG{l+m+mi}{3}\PYG{p}{)}
\PYG{g+go}{2.0}
\end{sphinxVerbatim}

\end{description}

\end{fulllineitems}


\sphinxstepscope


\chapter{Примеры использования}
\label{\detokenize{usage:id1}}\label{\detokenize{usage::doc}}
\begin{sphinxVerbatim}[commandchars=\\\{\}]
\PYG{k+kn}{from}\PYG{+w}{ }\PYG{n+nn}{calculator}\PYG{+w}{ }\PYG{k+kn}{import} \PYG{n}{add}\PYG{p}{,} \PYG{n}{subtract}
\PYG{n+nb}{print}\PYG{p}{(}\PYG{n}{add}\PYG{p}{(}\PYG{l+m+mi}{2}\PYG{p}{,} \PYG{l+m+mi}{3}\PYG{p}{)}\PYG{p}{)}  \PYG{c+c1}{\PYGZsh{} Выведет 5.0}
\PYG{n+nb}{print}\PYG{p}{(}\PYG{n}{subtract}\PYG{p}{(}\PYG{l+m+mi}{5}\PYG{p}{,} \PYG{l+m+mi}{3}\PYG{p}{)}\PYG{p}{)}  \PYG{c+c1}{\PYGZsh{} Выведет 2.0}
\end{sphinxVerbatim}


\renewcommand{\indexname}{Содержание модулей Python}
\begin{sphinxtheindex}
\let\bigletter\sphinxstyleindexlettergroup
\bigletter{c}
\item\relax\sphinxstyleindexentry{Calculator}\sphinxstyleindexpageref{Calculator:\detokenize{module-Calculator}}
\end{sphinxtheindex}

\renewcommand{\indexname}{Алфавитный указатель}
\printindex
\end{document}